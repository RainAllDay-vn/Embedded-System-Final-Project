\documentclass[12pt, a4paper]{article}

% GÓI NGÔN NGỮ VÀ FONT
\usepackage[utf8]{inputenc}
\usepackage[english]{babel}
\usepackage[T1]{fontenc}
\usepackage{helvet} 
\usepackage{color} % Thêm gói màu
\usepackage{xcolor}
\usepackage{tabularx} % Hỗ trợ bảng tự động chỉnh độ rộng cột
\usepackage{booktabs} % Giúp đường kẻ bảng chuyên nghiệp hơn
\usepackage{listings} % For code snippets
\usepackage{float}
\renewcommand{\arraystretch}{1.4} % Tăng khoảng cách giữa các dòng trong bảng

% CODE LISTING STYLE
\definecolor{codegreen}{rgb}{0,0.6,0}
\definecolor{codegray}{rgb}{0.5,0.5,0.5}
\definecolor{codepurple}{rgb}{0.58,0,0.82}
\definecolor{backcolour}{rgb}{0.95,0.95,0.92}

\lstdefinestyle{mystyle}{
    backgroundcolor=\color{backcolour},   
    commentstyle=\color{codegreen},
    keywordstyle=\color{magenta},
    numberstyle=\tiny\color{codegray},
    stringstyle=\color{codepurple},
    basicstyle=\ttfamily\footnotesize,
    breakatwhitespace=false,         
    breaklines=true,                 
    captionpos=b,                    
    keepspaces=true,                 
    numbers=left,                    
    numbersep=5pt,                  
    showspaces=false,                
    showstringspaces=false,
    showtabs=false,                  
    tabsize=2
}
\lstset{style=mystyle}

% GÓI BỐ CỤC VÀ HÌNH ẢNH
\usepackage[margin=1in]{geometry}
\usepackage{graphicx} 
\usepackage[hidelinks]{hyperref} 
\usepackage{booktabs} 
\usepackage{parskip} 
\usepackage{float}
\usepackage{fancyhdr} 
\usepackage{titlesec} 
\usepackage{longtable} % Cho bảng dài qua trang
\usepackage{array}

% TÙY CHỈNH TIÊU ĐỀ SECTION
\titleformat{\section}
  {\normalfont\Large\bfseries\color{blue!60!black}}
  {\thesection}{1em}{}
\titleformat{\subsection}
  {\normalfont\large\bfseries\color{black}}
  {\thesubsection}{1em}{}

% HỘP ĐỂ GIỮ CHỖ HÌNH ẢNH
\newcommand{\imageplaceholder}[2]{
  \begin{figure}[H]
    \centering
    \fbox{
      \parbox[c][#1][c]{0.8\textwidth}{
        \centering
        \vspace{10pt}
        \texttt{#2}
        \vspace{10pt}
      }
    }
    \caption{#2}
  \end{figure}
}

% --- BẮT ĐẦU TÀI LIỆU ---
\begin{document}

% --- BẮT ĐẦU TRANG BÌA MỚI ---
\begin{titlepage}
    \centering
    \pagestyle{empty} % <-- Đảm bảo trang này không có header/footer
    
    % 1. Thông tin trường
    {\Large \textbf{Hanoi University of Science and Technology}} \\
    \vspace{0.2cm}
    {\large School of Information and Communication Technology} \\
    
    \vspace{1.5cm}
    
    % 2. Logo
    \includegraphics[width=0.3\textwidth]{logo_hust.png} \\
    
    \vspace{1.5cm}
    
    % 3. Tiêu đề chính
    {\huge \textbf{Project Report: Game Tetris using STM32CubeIDE hardware and software}} \\
    \vspace{1cm}
    
    % 4. Thông tin môn học
    {\LARGE IT4210E - Embedded Systems} \\
    \vspace{0.5cm}
    
    \vfill % <-- Đẩy nội dung bên dưới xuống
    
    % 6. Thông tin nhóm
    {\Large \textbf{Group Members:}} \\
    \vspace{0.5cm}
    {\large
    Group 5 - Class: 161346\\
    Luong Ngoc Vu Long - 20235967\\
    Tran Sy Nguyen - 20235985\\
    Nguyen Vu Anh Khoa - 20235957
    } \\ 
    
    \vspace{2cm}
    
    % 7. Giảng viên
    {\Large \textbf{Lecturers:}} \\
    \vspace{0.5cm}
    {\large
    Prof. Ngô Lam Trung
    }
    
    \vfill % <-- Đẩy ngày tháng xuống dưới cùng
    
    % 8. Ngày tháng
    {\large \today}
    
\end{titlepage}

\tableofcontents
\newpage

\section{Introduction}

This project aims to replicate the classic arcade game "Tetris" on an embedded system platform. The primary goal is to demonstrate the integration of real-time operating systems (FreeRTOS), graphical user interfaces (TouchGFX), and hardware peripheral control (GPIO, Timers, Interrupts) on the STM32F429I-DISCO development board.

The system features a 240x320 pixel color display, a dedicated audio engine for background music and sound effects using PWM, and physical button controls for game interaction. The software architecture is designed using the Model-View-Presenter (MVP) pattern provided by TouchGFX, ensuring a clean separation between game logic and visual rendering.

\section{Hardware Design}

\subsection{Development Board Specifications}

\subsection{Peripheral Configuration}

\subsubsection{Display Subsystem (LTDC \& DMA2D)}

\subsubsection{Audio Subsystem (PWM)}

\subsubsection{Input Controls (GPIO \& EXTI)}

\section{Software Design}

\subsection{Software Architecture Overview}

\subsection{FreeRTOS Configuration}

\subsection{Game Logic (The Model)}

\subsubsection{Grid Representation}

\subsubsection{Game Loop (`tick`)}

\subsection{Audio Engine Implementation}

\subsection{User Interface Design}

\subsubsection{Main Menu}

\subsubsection{Game Screen}

\subsection{Input Debouncing}

\subsection{Memory Management}

\section{Results and Conclusion}

\subsection{Project Outcomes}

\subsection{Future Improvements}

\end{document}